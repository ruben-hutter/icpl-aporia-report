\section{Introduction}

%The purpose of this section is to give a quick introduction about the project and its objectives.
"Every program is open source if you know Assembly." - Unknown. But must this always be the case? What if we could write programs in a high level programming language like python and
interleave them with many different programs without changing input-output behavior making it (almost) impossible to reverse engineer the original program? The interleaving part of this problem has already been 
solved by Ali Ajorian (TODO Luca: Insert Aporia Paper reference here) but the first part was where this project came into play.
The goal was to create a compiler from the python subset $L_if$ to the aporia language $L_{cfi}$ allowing the original program to be obfuscated at a later point in time.
This objective was achieved by generating two new intermediary languages $L_{if}^{sc}$ and $L_{if}^{flat}$, each with their own restrictions and compiling the input program 
in three separate steps in order to finally reach the desired $L_{cfi}$ representation.


In the following report, we will give an overview of the technical background before we explain the implementation of the project. We will also discuss the difficulties that were 
encountered while coming up with solutions, evaluate the achieved outcomes, and reflect on the lessons learned during the course of this project.