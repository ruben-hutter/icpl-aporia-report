\section{Implementation}

\subsection{Overview}
This chapter goes over all of the compilation steps that were implemented to transform the input language $L_{if}$ into the output language $L_{cfi}$. 
The respective grammars of both languages are shown in Figure~\ref{bnf:lif} and Figure~\ref{bnf:lcfi}.
The compiler is implemented in Python and consists of three passes, each of which is responsible for specific transformations.
The following sections describe each pass in detail and show the transformations that are applied to the input program using the example program from Figure~\ref{code:lif}.

\begin{figure}[h]
	\centering
	\begin{bnf}[row{-} = {bg = gray9}]
		cmp ::= != // == // < // <= // > // >=;;
		exp ::= int // input\_int() // -exp
		| exp + exp // exp - exp
		| (exp) // var
		| True // False // exp and exp
		| exp or exp // not exp // exp cmp exp
		| exp if exp else exp;;
		stmt ::= print(exp) // exp // var = exp
		| if exp $\colon$ stmt$^+$ else $\colon$ stmt$^+$;;
		$L_{if}$ ::= stmt$^+$
	\end{bnf}
	\caption{Backus Naur Form Grammar of $L_{if}$}
	\label{bnf:lif}
\end{figure}

\begin{figure}[h]
	\centering
	\begin{bnf}[row{-} = {bg = gray9}]
		cmp ::= != // == // < // <= // > // >=;;
		bop ::= + // - // * // / // \%;;
		pred ::= true // false // bool\_var;;
		bool\_exp ::= pred // (bool\_exp) // !bool\_exp
		| bool\_exp \&\& bool\_exp
		| bool\_exp cmp bool\_exp;;
		num\_exp ::= number // num\_var // (num\_exp) // -num\_exp
		| num\_exp bop num\_exp;;
		inst ::= print("string",bool\_exp) // bool\_exp
		| print("string", num\_exp) // num\_exp
		| bool\_var $=$ bool\_exp
		| num\_var $=$ num\_exp;;
		stmt ::= (\$label)? // pred $\colon$ inst;;
		declar ::= bool bool\_var$^+$ // int num\_var$^+$ // int float\_var$^+$;;
		$L_{cfi}$ ::= declar$^*$ stmt$^*$
	\end{bnf}
	\caption{Backus Naur Form Grammar of $L_{cfi}$}
	\label{bnf:lcfi}
\end{figure}

\begin{figure}[h]
	\centering
	\lstinputlisting[language=python]{reportContent/code_examples/0_pres_example.py}
	\caption{$L_{if}$ Example Program Before Compilation}
	\label{code:lif}
\end{figure}

\subsection{First pass: Remove complex operands ($L_{if}^{sc}$)}


The first pass of the compiler lays an important foundation for each of following passes. It takes the $L_{if}$ AST of the original program as input and performins the following four transformations leading to the intermediary language $L_{if}^{sc}$ and its corresponding AST as output:

\begin{figure}[h!]
	\centering
	\begin{bnf}[row{-} = {bg = gray9}]
		cmp ::= != // == // < // <= // > // >=;;
		atm ::= True // False // var;;
		exp ::= int  // -exp // exp + exp 
		| exp - exp // (exp)
		| exp and exp // exp or exp 
		| not exp // exp cmp exp // atm;;
		stmt ::= print(exp) // exp // var = exp
		| if atm$\colon$ stmt$^+$;;
		$L_{sc}$ ::= stmt$^+$;;
	\end{bnf}
	\caption{Backus Naur Form Grammar of $L_{sc}$}
	\label{bnf:sc}
\end{figure}

\subsubsection{Remove top level expressions that have no effect:}
Any top level expression like a simple addition \texttt{1 + 2} or a boolean operation \texttt{True and False} is removed from the program because it does not contribute to any side effects. In the example program from Figure~\ref{code:lif}, this leads to line 2 being removed.
This transformation goes against the principle of Aporia, which wants to make the original program as complicated as possible but was necessary to simplify the implementation of the compiler.

\subsubsection{Create variables for complex conditions:}
The predicates in the statements of $L_{cfi}$ cannot be complex boolean expressions but only variables or the two boolean constants \texttt{true} and \texttt{false}.
Hence, if the compiler encounters a complex boolean expression as the condition of an if-statement, it creates a new boolean variable and assigns the complex expression to it. 
The compiler then replaces the complex expression with the new variable as the condition.
This transformation can be seen on lines 2 and 4 in Figure~\ref{code:sc}.

\subsubsection{Remove else branches:}
Each statement in $L_{cfi}$ can only have one branch, so the compiler must remove the else branch of if-statements.
In order to complete this step, the compiler assigns the negated condition of the if-statement to a new boolean variable.
This variable is then used as the condition of a second if-statement, which contains the else branch of the original statement as its body.
On line 3 of Figure~\ref{code:sc}, the assignment of the negation of the original condition can be seen which is then used as the condition on line 12 for the new if-statement representing the else branch of the original program.

\subsubsection{Turn if-expressions into if-statements:}
$L_{cfi}$ does not support if-expressions, so the compiler must transform them into if-statements with simple conditions and empty else branches. This is done in a similar fashion as in the exercise part of the seminar.
First, the condition is made atomic and the its negation assigned to a new variable as in the previous steps.
Then, two if-statements are created, one for the body and one for the orelse branch of the original if-expression.
Since top level expressions are ignored, the only context that if-expressions can occur in are print statements or variable assignments either as top level statements or to atomify a complex condition.
If the expression is part of a top level assignment, the variable name is given by the original program. In both other cases, the compiler generates a new variable name.
The result of the if-expression must thus always be assigned to the given variable name as the last statement of the new body in the new if-statement.
The whole transformation of an if-expression are shown in lines 5-10 of Figure~\ref{code:sc}.

\begin{figure}[h]
	\centering
	\lstinputlisting[language=python]{reportContent/code_examples/1_pres_example.py}
	\caption{$L_{if}^{sc}$ Representation After the First Pass of Example Program}
	\label{code:sc}
\end{figure}


\subsection{Second pass: Flatten and Make everything an if ($L_{if}^{flat}$)} \label{subsec:flatten}
\begin{figure}[h!]
	\centering
	\begin{bnf}[row{-} = {bg = gray9}]
		cmp ::= != // == // < // <= // > // >=;;
		atm ::= True // False // var;;
		exp ::= int  // -exp // exp + exp 
		| exp - exp // (exp)
		| exp and exp // exp or exp 
		| not exp // exp cmp exp // atm;;
		stmt ::= print(exp) // exp // var = exp;;
		if\_stmt ::= if atm$\colon$ stmt;;
		$L_{flat}$ ::= if\_stmt$^+$;;
	\end{bnf}
	\caption{Backus Naur Form Grammar of $L_{flat}$}
	\label{bnf:flat}
\end{figure}

The goal of this pass is to eliminate nested \texttt{if} statements, ensuring that all statements are wrapped in top-level \texttt{if} statements in the resulting intermediary language. We call this language $L_{flat}$, which inherits from $L_{sc}$ but restricts all \texttt{if} statements to be top-level only, as described by the BNF in Figure~\ref{bnf:flat}. Consequently, each \texttt{if} statement body now contains exactly one statement, which can not be an \texttt{if} statement, thereby preventing any form of nesting. The compiler takes an $L_{sc}$ AST as input and produces an $L_{flat}$ AST. Applying this pass to the sample code shown in Figure \ref{code:sc} yields the flattened version in Figure~\ref{code:flat}. The compiler of this pass makes the following transformations:

\subsubsection{Wrap every top level statement with if True} Any statement at the top level (i.e., not in an \texttt{if}) is enclosed in an \texttt{if True} block. This step ensures that every statement in the resulting code is preceded by \texttt{if}. We can see that lines 1-2 in Figure \ref{code:sc} transform to lines 3-4 in Figure \ref{code:flat}.

\subsubsection{Flatten nested if statements:}
When the compiler encounters an \texttt{if} statement nested within another \texttt{if}, it creates a temporary boolean variable that stores the conjunction of the outer and inner conditions. In Figure~\ref{code:sc}, for example, the nested \texttt{if} on line 7 is transformed into the conjunction on line 12 in Figure~\ref{code:flat}. The statements within the nested \texttt{if} block are then placed in a new \texttt{if} statement whose predicate is this temporary variable, as shown on lines 13–14 in Figure~\ref{code:flat}. By applying this process repeatedly, the compiler flattens all nested \texttt{if} statements without altering the program’s logical flow.

\begin{figure}[h!]
	\lstinputlisting[language=python]{reportContent/code_examples/2_pres_example.py}
	\caption{$L_{flat}$ After the Second Pass of Example Program}
	\label{code:flat}
\end{figure}


\subsection{Third pass: Select instructions ($L_{cfi}$)}

In this final stage, we compile $L_{flat}$ to $L_{cfi}$. The Backus-Naur Form (BNF) grammar of $L_{cfi}$ is presented in Figure~\ref{bnf:lcfi}, while the final compiled $L_{cfi}$ source code is shown in Figure~\ref{code:lcfi}. Since the abstract syntax of $L_{flat}$ closely resembles that of $L_{cfi}$, the primary task is to establish a one-to-one mapping for each syntax element. A notable difference between the two languages is that in $L_{cfi}$, the type of each variable must be explicitly declared in advance, similar to C-like languages. The responsibilities of this compiler include the following:

\subsubsection{Determine the Types of All Variables}
The source language $L_{if}$ supports only boolean and integer variables. Therefore, the compiler's task is limited to distinguishing between these two types. Each time the compiler encounters an assignment, it deduces the type of the right-hand side expression. Based on this deduction, it categorizes the left-hand side variable into either an integer set or a boolean set. Once all variable types have been identified, the compiler inserts their declarations at the beginning of the program, adhering to the $L_{cfi}$ syntax. These declarations can be observed in lines 1–2 of Figure~\ref{code:lcfi}.

\subsubsection{Translate Abstract Syntax Tree (AST) Objects}
This task involves translating each Python AST object into an equivalent $L_{cfi}$ AST object. The process is relatively straightforward, requiring a mapping of Python AST objects to corresponding $L_{cfi}$ objects. In most cases, the $L_{cfi}$ objects have the same arguments as their Python counterparts. However, an important distinction lies in the concrete syntax representation, as the string representations of the objects differ.

\begin{figure}[h!]
	\lstinputlisting{reportContent/code_examples/3_pres_example.spp}
	\caption{$L_{cfi}$ Representation After Final Pass of Example Program}
	\label{code:lcfi}
\end{figure}

\subsection{Testing Programs}

The task of a compiler is to translate a source language into a target language such that the behavior of the interpreted source language matches that of the interpreted compiled language. To test our compiler, we verified whether the output of the Python-interpreted source file was equal to the output of the compiled result interpreted by Aporia.
\\

We developed a command-line utility called \texttt{team\_run\_tests.py}, which takes a Python file and a pass number as input. The utility compiles the Python file incrementally, executing each pass until the specified pass number is reached. The results of the first two passes are interpreted using Python, while the final pass is interpreted by Aporia, which utilizes a binary called \texttt{spp}. For each pass, the utility checks whether the output of the interpreted source file matches the output of the interpreted compiled file. If the outputs are identical, it prints \texttt{SUCCESS}.
\\

Additionally, the utility offers the following options:
\begin{itemize}
	\item \texttt{--trace}: Prints the compiled source code for each pass.
	\item \texttt{--ast}: Prints the Abstract Syntax Tree (AST) for each pass.
\end{itemize}
\\

To integrate the \texttt{spp} binary provided by the Aporia project, we implemented a Python wrapper, as Aporia is written in C++. This process required writing a CMake build file for Aporia. The wrapper automates the execution of \texttt{cmake} and \texttt{make} to build the \texttt{spp} binary. It accepts $L_{cfi}$ source code as a string, writes it to a temporary file, and feeds this file into \texttt{spp}. The wrapper redirects \texttt{stdout} to capture the interpreted result, which is then returned to the caller. Consequently, the wrapper serves as a Python interface for interacting with \texttt{spp}.

